\documentclass[12pt,a4paper]{letter}
\usepackage[utf8x]{inputenc} \usepackage{ucs}
\usepackage{amsmath}
\usepackage{amsfonts}
\usepackage{amssymb}
\usepackage[T2A]{fontenc}
\usepackage[english,russian]{babel}
\address{Левицкий Дмитрий гр.8362} 
\signature{Левицкий Дмитрий} 
\begin{document} 
\begin{letter}{Зорин Кирилл} 
\opening{Здравствуйте,}
Я бы хотел спроектировать базу данных онлайн-магазина по продаже дисков компьютерных игр. Далее привожу \textit{примерное} описание.

Магазин будет заниматься продажей компьютерных дисков с играми через интернет с помощью платежной системы. Доставка покупателю возможна наземной почтой, курьером или авиапочтой (каждый вид доставки будет иметь свою стоимость и, возможно, скидки).

Каждый покупатель будет иметь собственный аккаунт, где хранятся его ФИО, логин, пароль, адрес электронной почты, полный домашний адрес, номер счета платежной системы, размер накопительной скидки и контактный телефон. У каждого покупателя есть счета (номера счетов), которые содержат информацию о способе, стоимости и дате отправки купленного покупателем товара, а также скидку на доставку. У каждой продажи есть свой уникальный номер. Покупатель может купить несколько дисков сразу и они будут отправлены единой посылкой в течение пары дней. Если этот же покупатель совершит еще одну покупку в день его предыдущей покупки, то этот товар будет оправлен в первой посылке. В противном случае, новый заказ будет выслан другой посылкой через несколько дней. 

Магазин обслуживает персонал. Продавцы занимаются оформлением счетов покупателей. Каждый сотрудник имеет собственный аккаунт, где хранятся его ФИО, логин, пароль, должность, адрес электронной почты, полный домашний адрес, номер счета платежной системы, оклад, процентная ставка к зарплате с продажи товара и контактный телефон.

У каждой игры указаны название, разработчик, издатель, дата выпуска, описание, количество дисков на складе и процент надбавки на цену покупки у поставщика для получения прибыли, а также цена последней поставки.
Магазин сотрудничает с поставщиками дисков. 

У каждого поставщика есть название, адрес электронной почты, адрес, номер счета и контактный телефон. Поставка товара идет партиями, каждая партия имеет свой номер. В партии указан какой товар, по какой цене, в каком количестве, когда и кем поставляется. 
Цена на диск складывается из цены последней поставки плюс процент для получения прибыли.
На покупки распространяются какие-нибудь скидки. Также каждый покупатель имеет индивидуальную накопительную скидку.
\closing{С уважением, Л.Д.} 
%\cc{Cclist} 
%\ps{adding a postscript} 
%\encl{list of enclosed material} 
\end{letter} 
\end{document}