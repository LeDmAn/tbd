\documentclass[12pt,a4paper]{article}
\usepackage[T2A]{fontenc}
\usepackage[utf8x]{inputenc}
\usepackage[english,russian]{babel}
\usepackage{ucs}
\usepackage{amsmath}
\usepackage{amsfonts}
\usepackage{amssymb}
\usepackage[left=2.00cm, right=1.50cm, top=1.00cm, bottom=2.00cm]{geometry}
\usepackage{indentfirst} %делать отступ в начале параграфа
\usepackage{enumerate}  %создание и автоматическая нумерация списков
\usepackage{tabularx}  %продвинутые таблицы
\usepackage[labelsep=endash, singlelinecheck=0]{caption} %Подпись. Разделитель - тире, формат по левому краю.
\captionsetup[table]{position=top, skip=0pt} %отбивка снизу подписи, 0pt
%\usepackage[unicode, colorlinks=true]{hyperref} %гиперссылки 
%\hypersetup{
%	pdftitle={},
%	pdfauthor={}
%}
%\usepackage{showkeys}  %раскомментировать, чтобы в документе были видны ссылки на литературу, рисунки и таблицы
%\usepackage[onehalfspacing]{setspace} %"умное" расстояние между строк - установить  1.5 интервала от нормального, эквивалентно 
%\usepackage[dvips]{graphicx} %разрешить включение PostScript-графики
%\graphicspath{{edgeimages/}} %относительный путь к каталогу с рисунками,это может быть мягкая ссылка

%\makeatletter
%\bibliographystyle{unsrt} %Стиль библиографических ссылок БибТеХа - нумеровать в порядке упоминания в тексте
%Заменяем библиографию с квадратных скобок на точку в списке литературы
%\renewcommand{\@biblabel}[1]{#1.}
%\makeatother

%\righthyphenmin=2 % Минимальное число символов при переносе - 2.
\newcommand{\tocsecindent}{\hspace{6.5mm}} %отступ для имени раздела в оглавлении
\numberwithin{equation}{section} %независимая нумерация формул внутри раздела
\numberwithin{table}{section}
\begin{document}
\begin{titlepage}
\newpage

\begin{center}
МИНОБРНАУКИ РОССИИ\\
Федеральное государственное бюджетное образовательное учреждение\\
высшего профессионального образования\\
“Санкт-Петербургский государственный электротехнический университет 
“ЛЭТИ” им. В.И.Ульянова (Ленина)”\\
(СПбГЭТУ)

\hrulefill
\end{center}

\begin{flushleft}
Факультет компьютерных технологий и информатики\\
\vspace{1em}
Специальность 090102\\
\vspace{1em}

Кафедра вычислительной техники
\end{flushleft}

\vspace{10em}


\begin{center}
\Large \textbf{Пояснительня записка \\ к курсовому проекту}
\end{center}

\begin{center}
\large\textit{\textbf{<<Проектирование базы данных PostgreSQL>>}}
\end{center}

\vspace{10em}


\begin{flushleft}
Выполнил Левицкий Д.\\
\vspace{1em}
Группа 8362\\
\vspace{1em}
Проверил Зорин К.
\end{flushleft}

\vspace{\fill}

\begin{center}
Санкт-Петербург\\
2012
\end{center}

\end{titlepage}
\setcounter{page}{2}
\tableofcontents
\input{Intro} %Введение
\addcontentsline{toc}{section}{\tocsecindent{Введение}} %Добавляем к содержанию, т.к. оно не нумеруется.

\input{Conclusion} %Заключение
\addcontentsline{toc}{section}{\tocsecindent{Заключение}}
\end{document}